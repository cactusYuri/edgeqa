\section{Experiments and Analyses}
\label{sec:experiments}

The primary goal of this paper is to release (i) a model-aware QA construction pipeline, (ii) two redistributable instantiations on CC-BY corpora, and (iii) a coverage-oriented evaluation suite that is directly reusable for IR and RAG research.
Accordingly, our experimental design emphasizes \emph{API-only feasibility}, \emph{model-specificity diagnostics}, and \emph{cost/coverage reporting}.

\subsection{Corpora and preprocessing}
\label{sec:corpora}
We instantiate EdgeQA and EdgeCoverBench on two open corpora with clear redistribution terms (CC BY 4.0) and strong structure for deterministic unit extraction:
\begin{enumerate}[leftmargin=*]
  \item \textbf{Open Logic Project (OLP):} a LaTeX-based logic textbook.
  \item \textbf{OpenStax University Physics (OSP):} an open textbook series with stable chapter/section structure.
\end{enumerate}

\begin{table}[t]
\centering
\small
\begin{tabular}{@{}p{1.4cm}p{2.7cm}p{1.3cm}p{2.8cm}p{1.2cm}@{}}
\toprule
\textbf{Corpus} & \textbf{Source} & \textbf{License} & \textbf{Primary structured units} & \textbf{Evidence} \\
\midrule
OLP & Open Logic Project (LaTeX) & CC BY 4.0 & Def/Theorem/Lemma/Proof/Example & raw text \\
OSP & OpenStax Univ. Physics (HTML) & CC BY 4.0 & Definitions, laws, equations, examples & raw text \\
\bottomrule
\end{tabular}
\caption{Core corpora in this release. Because both are CC BY 4.0, we redistribute raw evidence text with attribution (Sec.~\ref{sec:availability}).}
\label{tab:corpora}
\end{table}

\paragraph{Segmentation and stable identifiers.}
For each corpus, we segment into passages (typically paragraphs) with stable identifiers.
For OLP, we respect LaTeX environments and section paths.
For OSP, we parse HTML into sections and paragraphs and remove navigation boilerplate.
To bound API cost, we cap each evidence span to at most $L_e$ tokens (default $L_e{=}512$) by selecting the minimal supporting sentences, and we split very long paragraphs into sub-passages.
We deduplicate passages using exact hashing and a near-duplicate filter.

\paragraph{Construction vs. evaluation partitions.}
To reduce leakage in coverage diagnostics, we split documents into \emph{construction} and \emph{evaluation} partitions by chapter/section (default 80/20).
EdgeQA uses the construction partition for dataset synthesis, while EdgeCoverBench and coverage reporting can be computed on either the full corpus or the evaluation partition depending on the intended use.

\subsection{Knowledge unit extraction (deterministic-first)}
We report both \emph{structured-unit} coverage (preferred when available) and \emph{fallback} unit coverage (atomic facts).

\paragraph{Structured units (low noise).}
\textbf{OLP units} are parsed directly from LaTeX environments, with unit IDs as \texttt{(doc\_id, env\_type, env\_counter)}.
\textbf{OSP units} are extracted from (i) definition/key-term blocks, (ii) named laws and principles, (iii) displayed equations with surrounding explanation, and (iv) worked examples.

\paragraph{Fallback units (API-based atomic facts).}
When structure is absent or when we want finer-grained accounting, we decompose passages into atomic facts using the same API model stack as the rest of the pipeline (Sec.~\ref{sec:api_config}).
To quantify extractor noise, we recommend running two prompt variants and reporting UnitCov uncertainty bands (mean\,$\pm$\,std).

\subsection{API-only model stack and token budgets}
\label{sec:api_config}
EdgeQA assumes black-box access to a target model $M$ (no token logprobs required).
In this release, we use \textsc{DeepSeek-V3.2} via API in a single mode:
\textbf{\texttt{deepseek-chat}} (non-thinking) for \emph{all} high-volume steps.
We optionally support a stricter ``thinking'' verifier model for a small audit slice, but we disable it in the released instantiations.
We cache all API inputs/outputs and record model name, access date, decoding parameters, prompt template version, and a content hash for reproducibility.

\paragraph{Token budgets.}
We target a non-thinking budget of \textbf{500M tokens} for \texttt{deepseek-chat} (expandable).
If using a thinking verifier, we recommend capping it (e.g., \textbf{20M tokens}) and reserving it for borderline near-miss validation and a small audit slice.

\begin{table}[t]
\centering
\small
\begin{tabular}{@{}p{2.1cm}p{2.0cm}p{0.9cm}p{0.9cm}p{1.4cm}p{1.3cm}@{}}
\toprule
\textbf{Step} & \textbf{Model} & $T$ & $p$ & \textbf{Max out} & \textbf{Calls} \\
\midrule
QA generation & \texttt{deepseek-chat} & 0.7 & 0.9 & 256 & 1 / passage \\
Closed-book answer & \texttt{deepseek-chat} & 0.0 & 1.0 & 64 & 1 / QA \\
Sampling inconsistency & \texttt{deepseek-chat} & 0.8 & 0.95 & 64 & $m{=}4$ / QA$^\dagger$ \\
Paraphrase gen & \texttt{deepseek-chat} & 0.8 & 0.95 & 192 & 1 / QA$^\dagger$ \\
Paraphrase answers & \texttt{deepseek-chat} & 0.0 & 1.0 & 64 & $k{=}2$ / QA$^\dagger$ \\
Fast verification & \texttt{deepseek-chat} & 0.0 & 1.0 & 64 & 1 / QA$^\ddagger$ \\
Strict verification (optional) & \texttt{deepseek-chat} & 0.0 & 1.0 & 128 & \small subset \\
\bottomrule
\end{tabular}
\caption{Default API configuration. $T$=temperature; $p$=top-$p$. $^\dagger$Applied only to candidates that fail closed-book and pass basic quality gates. $^\ddagger$Applied only to candidates that pass the edge score threshold and are considered for the final pool.}
\label{tab:api_config}
\end{table}

\subsection{Default EdgeQA configuration for this release}
Table~\ref{tab:default_config} lists a concrete, token-budget-aware configuration that fits within the stated DeepSeek budgets.
We report results as coverage curves over multiple final budgets $N$.

\begin{table}[t]
\centering
\small
\begin{tabular}{@{}p{3.6cm}p{4.8cm}@{}}
\toprule
\textbf{Component} & \textbf{Default setting} \\
\midrule
Candidate passages $|\mathcal{P}_0|$ & 20k per corpus (salience + topic-stratified sampling) \\
QA per passage & 2 single-hop + 0--1 multi-hop (rate 0.25) \\
Evidence span cap & $L_e{=}512$ tokens (minimal supporting sentences) \\
Sampling inconsistency & $m{=}4$ stochastic samples for CB-fail candidates \\
Paraphrases & $k{=}2$ meaning-preserving paraphrases for CB-fail candidates \\
Edge score threshold & $\tau{=}0.60$ on $U(x)$ (Eq.~\ref{eq:edge_score}) \\
Fast verifier threshold & entailment score $\ge 0.8$ (LLM judge) \\
Selection objective weights & $(\lambda_d,\lambda_u,\lambda_r,\lambda_{dup})=(1,1,0.5,0.2)$ \\
Unknownness purity constraint & $\rho{=}0.50$ (optional constrained selection) \\
Final QA budgets & $N\in\{1k,2k,5k,10k\}$ per corpus \\
\bottomrule
\end{tabular}
\caption{Concrete configuration used for the released instantiations. All thresholds and budgets are exposed in configuration files and are reused across corpora unless stated otherwise.}
\label{tab:default_config}
\end{table}

\subsection{Cross-model evaluation (model-aware diagnostics)}
To test whether ``edge'' sets are genuinely model-aware rather than universally hard, we evaluate a \emph{different} API model on the released \texttt{deepseek-chat}-targeted EdgeQA exports.
We use \texttt{qwen-plus} (non-thinking) via batch inference and report closed-book accuracy, evidence-conditioned accuracy, and unknownness purity (Table~\ref{tab:cross_eval_qwen}).
At $N{=}10k$, \texttt{qwen-plus} reaches $\approx$0.69 context accuracy on both corpora but has unknownness purity $\approx$0.49--0.51, indicating that the released sets are \emph{not} universally ``unknown'' across models.
A full 2\,$\times$\,2 matrix (constructing separate EdgeQA sets for each target model) is left for future work due to additional API cost.

\begin{table}[t]
\centering
\small
\begin{tabular}{@{}lrrrr@{}}
\toprule
\textbf{Corpus} & \textbf{N} & CB Acc. & Ctx Acc. & Unk. purity \\
\midrule
OLP & 1k  & 0.260 & 0.649 & 0.438 \\
OLP & 2k  & 0.238 & 0.645 & 0.450 \\
OLP & 5k  & 0.206 & 0.662 & 0.491 \\
OLP & 10k & 0.206 & 0.688 & 0.514 \\
OSP & 1k  & 0.265 & 0.681 & 0.463 \\
OSP & 2k  & 0.284 & 0.680 & 0.441 \\
OSP & 5k  & 0.242 & 0.687 & 0.480 \\
OSP & 10k & 0.224 & 0.685 & 0.492 \\
\bottomrule
\end{tabular}
\caption{Cross-model evaluation of the released EdgeQA exports using \texttt{qwen-plus} in closed-book (CB) and evidence-conditioned (Ctx) settings. Unk. purity is the fraction where CB is incorrect but Ctx is correct.}
\label{tab:cross_eval_qwen}
\end{table}



\paragraph{Breakdown by hop.}
Multi-hop items are systematically less ``unknown'' to \texttt{qwen-plus}.
At $N{=}10k$, unknownness purity drops from $\approx$0.51 to $\approx$0.44 on OSP and from $\approx$0.54 to $\approx$0.44 on OLP when moving from single-hop to multi-hop, while closed-book accuracy increases.
This suggests that some multi-hop questions are easier to answer from parametric knowledge (less useful as ``unknown'' tests), and may require stronger evidence-conditioning or stricter unknownness constraints to remain model-specific.

\paragraph{Cheap retargeting and pool-only baselines.}
We additionally study (i) cheap retargeting by re-scoring unknownness on the same candidate pool for a different model and (ii) cost-free pool-only selection baselines; we include these in Appendix~\ref{app:extra_results} due to space.

\subsection{IR export and retrieval baselines}
\label{sec:ir_export}
Each grounded QA instance defines a query $x$ and a set of relevant evidence passages $E$.
We therefore export each instantiation as a BEIR-style test collection~\citep{thakur2021beir}:
\texttt{queries.jsonl} (questions), \texttt{corpus.jsonl} (passages), and \texttt{qrels} (evidence passages marked relevant).
For multi-hop items, we mark the union of evidence passages as relevant.

We report standard retrieval metrics, focusing on Recall@$k$ (critical for RAG) and ranking metrics such as nDCG@10.
We include representative sparse/dense baselines:
BM25, a lightweight dense retriever (sentence-transformer MiniLM), and a simple hybrid BM25+dense reranker.

\subsection{Metrics and analyses}
For each corpus and method, we report:
\begin{enumerate}[leftmargin=*]
  \item \textbf{Unknownness purity:} fraction where $M$ fails closed-book but succeeds with evidence.
  \item \textbf{Brittleness:} sampling entropy and pairwise paraphrase disagreement.
  \item \textbf{Coverage curves:} DocCov/UnitCov/ReasonCov as a function of QA budget $N$, plus marginal gain curves.
  \item \textbf{Cost-aware curves:} DocCov/UnitCov as a function of API token budget (Sec.~\ref{sec:api_config}).
  \item \textbf{IR metrics:} Recall@$\{5,10,20\}$, nDCG@10, and MRR@10 for the exported retrieval collections (Sec.~\ref{sec:ir_export}).
  \item \textbf{EdgeCoverBench robustness:} paraphrase accuracy, near-miss false positive rate, and risk--coverage curves under abstention (Sec.~\ref{sec:edgecoverbench}).
  \item \textbf{Quality gates:} verification pass rate, ambiguity rejection rate, and near-duplicate rate.
\end{enumerate}

\paragraph{Human audit.}
To mitigate LLM-as-a-judge bias (cf.\ LLM-based judging protocols~\citep{zheng2023mtbench}), we include a targeted manual audit for each corpus (default 100 items for EdgeQA and 100 for EdgeCoverBench):
answer correctness, evidence support, ambiguity, paraphrase meaning preservation, and (for EdgeCoverBench) the validity of near-miss/unanswerable labels.
Across both corpora, the audit indicates high data quality: for EdgeQA, answer correctness is 1.00 on both corpora with evidence support 1.00 (OSP) and 0.98 (OLP) and $\le$1\% ambiguity; for EdgeCoverBench, paraphrase meaning preservation and near-miss label validity are 1.00 (Table~\ref{tab:human_audit_edgeqa}, Table~\ref{tab:human_audit_ecb}).
\begin{table}[t]
\centering
\small
\begin{tabular}{@{}lrrrr@{}}
\toprule
\textbf{Corpus} & \textbf{N} & Ans. correct & Evidence support & Ambiguous \\
\midrule
OLP & 100 & 1.00 & 0.98 & 0.01 \\
OSP & 100 & 1.00 & 1.00 & 0.00 \\
\bottomrule
\end{tabular}
\caption{Human audit results for EdgeQA (C2). Entries are fractions over a 100-item stratified sample per corpus.}
\label{tab:human_audit_edgeqa}
\end{table}


\begin{table}[t]
\centering
\small
\begin{tabular}{@{}lrrrr@{}}
\toprule
\textbf{Corpus} & \textbf{N (can/para/nm)} & Can. Ans. & Para. meaning & Near-miss valid \\
\midrule
OLP & 40/40/20 & 1.00 & 1.00 & 1.00 \\
OSP & 40/40/20 & 1.00 & 1.00 & 1.00 \\
\bottomrule
\end{tabular}
\caption{Human audit results for EdgeCoverBench (C2). Can. Ans. is answer correctness on canonical items; Para. meaning is meaning preservation on paraphrase items; Near-miss valid is whether the near-miss label is valid (unanswerable/contradicted as intended).}
\label{tab:human_audit_ecb}
\end{table}



\subsection{Released instantiation results (DeepSeek-V3.2 chat)}
\label{sec:instantiation_results}
We report coverage curves and budgeted exports for the released \texttt{deepseek-chat} instantiations on OLP and OSP.
Table~\ref{tab:coverage_main} summarizes key coverage/composition numbers across budgets, and Figure~\ref{fig:coverage_curves} visualizes DocCov/UnitCov.
Appendix~\ref{app:extra_results} reports additional analyses (brittleness distributions, token breakdowns, pool-only baselines, retrieval baselines, and risk--coverage curves).

\begin{table}[t]
\centering
\small
\begin{tabular}{@{}lrrrr@{}}
\toprule
\textbf{Corpus} & \textbf{N} & \textbf{DocCov} & \textbf{UnitCov} & \textbf{Multi-hop} \\
\midrule
OLP & 1k & 0.224 & 0.311 & 1.000 \\
OLP & 2k & 0.383 & 0.470 & 0.714 \\
OLP & 5k & 0.744 & 0.690 & 0.388 \\
OLP & 10k & 0.775 & 0.690 & 0.251 \\
OSP & 1k & 0.109 & 0.231 & 0.941 \\
OSP & 2k & 0.203 & 0.309 & 0.812 \\
OSP & 5k & 0.396 & 0.456 & 0.427 \\
OSP & 10k & 0.676 & 0.456 & 0.240 \\
\bottomrule
\end{tabular}
\caption{Coverage and multi-hop composition for the released DeepSeek-V3.2 (chat) instantiations. DocCov/UnitCov are computed over corpus passages/units (Sec.~\ref{sec:unitcov}).}
\label{tab:coverage_main}
\end{table}


\begin{figure}[t]
\centering
\includegraphics[width=\linewidth]{figures/coverage_curves.pdf}
\caption{Coverage curves for the released instantiations as a function of QA budget $N$. DocCov measures the fraction of corpus passages exercised by at least one evidence set; UnitCov measures the fraction of extracted structured units covered.}
\label{fig:coverage_curves}
\end{figure}
\begin{table}[t]
\centering
\small
\begin{tabular}{@{}lrrrrr@{}}
\toprule
\textbf{Corpus} & \textbf{Rows} & Canon. & Para. & Near-miss & Unans. \\
\midrule
OLP & 10052 & 3187 & 6370 & 495 & 444 \\
OSP & 6575 & 2089 & 4176 & 310 & 234 \\
\bottomrule
\end{tabular}
\caption{EdgeCoverBench composition for the released corpora. Near-miss items include both contradicted and unanswerable variants.}
\label{tab:ecb_comp}
\end{table}


\paragraph{EdgeCoverBench baseline (\texttt{qwen-plus}).}
We evaluate \texttt{qwen-plus} on EdgeCoverBench in an evidence-conditioned setting with optional abstention.
Table~\ref{tab:ecb_qwen} reports paraphrase robustness and near-miss false positive rates, demonstrating that near-miss counterfactuals induce substantial false confidence under a standard API baseline.
Appendix~\ref{app:extra_results} (Figure~\ref{fig:ecb_risk_coverage_qwen}) reports risk--coverage curves under abstention by sweeping a confidence threshold.
\begin{table}[t]
\centering
\small
\begin{tabular}{@{}lrrrrrr@{}}
\toprule
\textbf{Corpus} & \textbf{Rows} & Canon Acc. & Para Acc. & Para Cons. & Near-miss Acc. & Near-miss FPR \\
\midrule
OLP & 10052 & 0.636 & 0.606 & 0.489 & 0.548 & 0.453 \\
OSP & 6575  & 0.686 & 0.659 & 0.662 & 0.436 & 0.565 \\
\midrule
ALL & 16627 & 0.656 & 0.627 & 0.558 & 0.504 & 0.496 \\
\bottomrule
\end{tabular}
\caption{EdgeCoverBench robustness results using \texttt{qwen-plus}. Para Cons. is exact paraphrase-consistency among pairs where both the canonical and paraphrase are non-abstained. Near-miss FPR is the fraction of near-miss items the model answered (non-empty, non-abstain).}
\label{tab:ecb_qwen}
\end{table}



\subsection{Reproducibility and configuration files}
The released repository includes an end-to-end pipeline (ingest $\rightarrow$ segment $\rightarrow$ mine $\rightarrow$ generate $\rightarrow$ score/filter $\rightarrow$ units $\rightarrow$ select $\rightarrow$ evaluate) and configuration files for every table/figure.
Because API calls are expensive and non-deterministic across time, we also release cached API responses for the reported experiments and provide scripts to re-run calls when an API key is available.
