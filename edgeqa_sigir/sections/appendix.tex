\section{Prompt templates (abbreviated)}
\label{app:prompts}
We include simplified prompt sketches to facilitate reimplementation.

\paragraph{QA generation (single-hop).}
\begin{quote}\small
Given the passage below, write \textbf{two} questions that can be answered \emph{only} using the passage. For each question, output a JSON object with:
(i) \texttt{question}, (ii) \texttt{answer} (short, unambiguous), (iii) \texttt{evidence\_span} (minimal supporting sentences or character offsets), and (iv) \texttt{reason\_type} from a small taxonomy.
Avoid yes/no questions and avoid copying long phrases from the evidence span.
\\
\textbf{Passage:} \dots
\end{quote}

\paragraph{Paraphrase generation (meaning-preserving).}
\begin{quote}\small
Rewrite the question in \textbf{two} meaning-preserving ways. Keep all constraints (numbers, quantifiers, conditions) unchanged. Do not add new information. Return JSON list of paraphrases.
\end{quote}

\paragraph{Fast verification (supported-by-evidence).}
\begin{quote}\small
Decide whether the proposed answer is fully supported by the evidence span.
Return one of: \texttt{ENTAILED}, \texttt{CONTRADICTED}, \texttt{NOT\_ENOUGH\_INFO}.
\\
\textbf{Question:} \dots\\
\textbf{Answer:} \dots\\
\textbf{Evidence span:} \dots
\end{quote}

\paragraph{Near-miss counterfactual generation.}
\begin{quote}\small
Modify the canonical question by changing \textbf{exactly one} minimal constraint (e.g., flip a quantifier, change a single numeric threshold, remove one condition) such that the modified question becomes unanswerable from the same evidence or contradicted by it.
Return JSON with the modified question and label (\texttt{UNANSWERABLE} or \texttt{CONTRADICTED}).
\end{quote}

\paragraph{Atomic fact decomposition (fallback units).}
\begin{quote}\small
Break the passage into a list of atomic, independently checkable facts.
Each fact should contain a single relation/event with minimal qualifiers.
Return as a JSON list of strings.
\end{quote}

\section{Concrete configuration (DeepSeek-V3.2)}
\label{app:config}
To make the resource usable out-of-the-box, we provide configuration files that specify models, decoding, thresholds, and budgets.
Below is an abbreviated example (YAML-style) matching Table~\ref{tab:api_config} and Table~\ref{tab:default_config}:

\begin{verbatim}
models:
  generator: deepseek-chat
  target: deepseek-chat
  verifier_fast: deepseek-chat
  verifier_strict: deepseek-reasoner   # limited usage

budgets:
  token_budget_chat: 500000000         # non-thinking
  token_budget_reasoner: 20000000      # thinking (limited)
  passage_candidates: 20000            # |P0|
  qa_per_passage: 2
  multi_hop_rate: 0.25
  final_N: [1000, 2000, 5000, 10000]

decoding:
  gen:          {temperature: 0.7, top_p: 0.9,  max_tokens: 256}
  closed_book:  {temperature: 0.0, top_p: 1.0,  max_tokens: 64}
  sample:       {temperature: 0.8, top_p: 0.95, max_tokens: 64, n: 4}
  paraphrase:   {temperature: 0.8, top_p: 0.95, max_tokens: 192, k: 2}
  verify:       {temperature: 0.0, top_p: 1.0,  max_tokens: 64}

thresholds:
  edge_tau: 0.60
  verifier_entailment_tau: 0.80
  max_evidence_tokens: 512

selection:
  lambdas: {doc: 1.0, unit: 1.0, reason: 0.5, redundancy: 0.2}
  unknownness_min_frac: 0.50
\end{verbatim}

\section{Cost accounting}
\label{app:cost}
For each candidate QA, EdgeQA typically requires:
(i) one evidence-grounded generation call,
(ii) one closed-book answer call,
and, for candidates that fail closed-book and pass basic gates,
(iii) $m$ sampling calls and (iv) one paraphrase-generation call plus $k$ paraphrase-answer calls.
We report total input/output tokens and provide cost--coverage curves (Sec.~\ref{sec:experiments}) so that users can compare construction methods under matched budgets.

\section{Additional experimental results}
\label{app:extra_results}
We collect secondary analyses and larger tables/figures here to keep the main body within the SIGIR Resources page limits.
All artifacts are reproducible from the released cached outputs and configuration files.

\paragraph{Pool-only baselines.}
\begin{table}[t]
\centering
\small
\begin{tabular}{@{}llrr@{}}
\toprule
\textbf{Corpus} & \textbf{Baseline} & \textbf{DocCov} & \textbf{UnitCov} \\
\midrule
OLP & EdgeQA & 0.775 & 0.690 \\
OLP & Coverage-first & 0.775 & 0.690 \\
OLP & Unit-first & 0.773 & 0.690 \\
OLP & Long-tail & 0.768 & 0.680 \\
OLP & Paraphrase-only & 0.770 & 0.684 \\
OLP & Random & 0.772 & 0.688 \\
OSP & EdgeQA & 0.676 & 0.456 \\
OSP & Coverage-first & 0.676 & 0.441 \\
OSP & Unit-first & 0.506 & 0.456 \\
OSP & Long-tail & 0.416 & 0.176 \\
OSP & Paraphrase-only & 0.444 & 0.253 \\
OSP & Random & 0.488 & 0.354 \\
\bottomrule
\end{tabular}
\caption{Selection baseline coverage at $N=10k$ using the same candidate pool (no additional API calls).}
\label{tab:selection_baselines}
\end{table}


\begin{figure}[t]
\centering
\includegraphics[width=\linewidth]{figures/selection_baselines.pdf}
\caption{DocCov curves for cost-free selection baselines on the shared EdgeQA candidate pool.}
\label{fig:selection_baselines}
\end{figure}

\paragraph{Cheap retargeting from the same pool.}
\begin{table}[t]
\centering
\small
\begin{tabular}{@{}lrrrr@{}}
\toprule
\textbf{Corpus} & \textbf{N} & \textbf{DocCov} & \textbf{UnitCov} & \textbf{Unk. frac (qwen)} \\
\midrule
OLP & 10k & 0.775 & 0.690 & 0.507 \\
OSP & 10k & 0.647 & 0.456 & 0.574 \\
\bottomrule
\end{tabular}
\caption{Retargeting selection to \texttt{qwen-plus} by re-scoring unknownness on the existing candidate pool (no new QA generation). Unk. frac is the fraction where \texttt{qwen-plus} fails closed-book but succeeds with evidence (unknownness purity).}
\label{tab:retarget_qwen_pool}
\end{table}



\paragraph{Retrieval baselines on BEIR-style exports.}
\begin{table}[t]
\centering
\small
\begin{tabular}{@{}lllrrrr@{}}
\toprule
\textbf{Corpus} & \textbf{N} & \textbf{Method} & R@5 & R@10 & R@20 & nDCG@10 \\
\midrule
OLP & 1k & BM25 & 0.468 & 0.541 & 0.609 & 0.486 \\
OLP & 1k & Dense & 0.462 & 0.547 & 0.644 & 0.472 \\
OLP & 1k & Hybrid & 0.523 & 0.608 & 0.679 & 0.540 \\
OLP & 2k & BM25 & 0.491 & 0.561 & 0.622 & 0.484 \\
OLP & 2k & Dense & 0.497 & 0.578 & 0.661 & 0.481 \\
OLP & 2k & Hybrid & 0.555 & 0.628 & 0.691 & 0.544 \\
OLP & 5k & BM25 & 0.507 & 0.576 & 0.637 & 0.476 \\
OLP & 5k & Dense & 0.530 & 0.615 & 0.688 & 0.489 \\
OLP & 5k & Hybrid & 0.580 & 0.653 & 0.713 & 0.545 \\
OLP & 10k & BM25 & 0.550 & 0.624 & 0.684 & 0.507 \\
OLP & 10k & Dense & 0.570 & 0.654 & 0.725 & 0.513 \\
OLP & 10k & Hybrid & 0.627 & 0.696 & 0.752 & 0.577 \\
OSP & 1k & BM25 & 0.441 & 0.495 & 0.536 & 0.432 \\
OSP & 1k & Dense & 0.480 & 0.554 & 0.628 & 0.467 \\
OSP & 1k & Hybrid & 0.475 & 0.540 & 0.580 & 0.468 \\
OSP & 2k & BM25 & 0.433 & 0.495 & 0.535 & 0.433 \\
OSP & 2k & Dense & 0.481 & 0.565 & 0.644 & 0.470 \\
OSP & 2k & Hybrid & 0.479 & 0.543 & 0.580 & 0.473 \\
OSP & 5k & BM25 & 0.517 & 0.565 & 0.599 & 0.495 \\
OSP & 5k & Dense & 0.574 & 0.640 & 0.699 & 0.537 \\
OSP & 5k & Hybrid & 0.566 & 0.611 & 0.640 & 0.540 \\
OSP & 10k & BM25 & 0.535 & 0.586 & 0.623 & 0.497 \\
OSP & 10k & Dense & 0.611 & 0.672 & 0.728 & 0.555 \\
OSP & 10k & Hybrid & 0.593 & 0.642 & 0.674 & 0.551 \\
\bottomrule
\end{tabular}
\caption{Retrieval results on the BEIR-style exports of EdgeQA (questions as queries; evidence passages as relevant). Dense uses \texttt{sentence-transformers/all-MiniLM-L6-v2}.}
\label{tab:bm25}
\end{table}


\paragraph{Brittleness and token accounting.}
\begin{figure}[t]
\centering
\includegraphics[width=\linewidth]{figures/brittleness_cdfs.pdf}
\caption{Brittleness distributions for the released instantiations at $N{=}10k$ per corpus. We report sampling entropy ($H_s$) and paraphrase disagreement ($1-A_p$), and stratify by single-hop vs.\ multi-hop items.}
\label{fig:brittleness_cdfs}
\end{figure}

\begin{figure}[t]
\centering
\includegraphics[width=\linewidth]{figures/token_breakdown.pdf}
\caption{Total token usage across all cached API calls for the released instantiations (EdgeQA + EdgeCoverBench).}
\label{fig:token_breakdown}
\end{figure}

\paragraph{EdgeCoverBench risk--coverage curves (abstention).}
\begin{figure}[t]
\centering
\includegraphics[width=\linewidth]{figures/edgecoverbench_risk_coverage_qwen_plus.pdf}
\caption{Risk--coverage curves for \texttt{qwen-plus} under abstention on EdgeCoverBench. We sweep a confidence threshold and report risk ($1-\textsc{Acc}$) against coverage (fraction answered).}
\label{fig:ecb_risk_coverage_qwen}
\end{figure}
