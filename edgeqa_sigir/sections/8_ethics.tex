\section{Ethical Considerations}
\label{sec:ethics}

\paragraph{Data sources and licensing.}
EdgeQA is designed to operate on openly available corpora.
We recommend selecting sources with clear redistribution rights and recording per-document license metadata.
When redistribution is not permitted, we release only derived QA artifacts and evidence pointers plus scripts to rehydrate evidence from upstream sources.

\paragraph{Privacy and personally identifiable information (PII).}
If the corpus contains PII, it should be filtered before QA generation.
We recommend automatic PII detection and conservative removal policies.

\paragraph{Potential misuse.}
Because EdgeQA is model-aware, it can be used to systematically discover a target model's blind spots.
This can benefit evaluation and safety work, but it could also be misused for targeted model attacks.
We recommend responsible disclosure of high-risk failure modes and, when appropriate, rate-limiting or controlled access to high-risk subsets.

\paragraph{Sensitive domains.}
When applying EdgeQA to biomedical or legal corpora, one should avoid releasing items that look like personalized medical or legal advice.
We recommend safety filtering for dosage/treatment plans, jurisdiction-specific legal advice prompts, and other potentially harmful content.
Evaluation should clearly distinguish between model behavior and evidence-supported ground truth to avoid reinforcing hallucinations.
